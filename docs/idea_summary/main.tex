\documentclass[11pt,a4paper]{article}

% Bibliography package
\usepackage[backend=biber,
style=authoryear,
date=year,
block=space,
maxbibnames=10,
doi=false,isbn=false,url=true,eprint=false]
{biblatex}

\AtEveryBibitem{%
	\ifentrytype{article}{%
		\clearfield{url}%
		\clearfield{urldate}%
		\clearfield{archivePrefix}%
		\clearfield{arxivId}%
	}{%
		\ifentrytype{misc}{%
			\clearfield{url}%
			\clearfield{urldate}%
			\clearfield{archivePrefix}%
			\clearfield{arxivId}%
		}{%
			\ifentrytype{incollection}{%
				\clearfield{url}%
				\clearfield{urldate}%
				\clearfield{archivePrefix}%
				\clearfield{arxivId}%
			}{%
			}%
		}%
	}%
}%

\addbibresource{../Bibliographies/202012.bib}


% Set languange:
\usepackage[utf8]{inputenc}
\usepackage[UKenglish]{babel}
\usepackage{csquotes}

% Math Packages
\usepackage{amsmath}
\usepackage{amsfonts}
\usepackage{amssymb}

% Page layout packages
\usepackage[margin=0.8in, top=0.5in]{geometry}
\usepackage[onehalfspacing]{setspace}
%\usepackage{fancyhdr}
%\setlength{\headheight}{14pt}

% Figure packages
\usepackage{graphicx}
\usepackage{color}
\usepackage{xcolor}

% Table packages
\usepackage[justification=centering]{caption}
\usepackage{subcaption}
\usepackage{longtable}
\usepackage{tabularx}
\usepackage{booktabs} % For Stata Tables
%\usepackage{dcolumn} % to align decimals in tables
\usepackage{siunitx} % to align decimals in tables
\usepackage{rotating}
\usepackage{bigstrut}
\usepackage{multirow}
\usepackage{wrapfig}

% Other packages
\usepackage[pdftex,colorlinks=true]{hyperref}
\hypersetup{pdfauthor={Nicol\'as A. Lillo B. and Felipe Valencia C.},%
	pdftitle={Research Idea - Capitanias},%
	citecolor=blue,%
	urlcolor=purple,%
	filecolor=cyan}
%\usepackage[raggedright]{titlesec} % Avoid hyphenation in section titles

\renewcommand{\and}{ \& }

% Declare title info:
\title{Colonial Autonomy and State-Capacity: Evidence from Brazilian Capitan\'ias}
\date{December 2020}
\author{Nicol\'as A. Lillo B.\thanks{Department of Economics, Pontificia Universidad Javeriana, \emph{Corresponding Author: \href{mailto:nicolas_lillob@javeriana.edu.co}{nicolas\_lillob@javeriana.edu.co}}} \and Felipe Valencia C.\thanks{Vancouver School of Economics, University of British Columbia and IZA}}

\begin{document}
	\maketitle
	
	\begin{itemize}
		\item We plan to answer the question in \textcite{Iyer2010}: what had better (less worse) consequences, direct or indirect rule by a colonizing power.
		\item We will exploit quasi-random variation stemming from the way that Brazil was colonized. When Brazil was subdivided in \emph{capitan\'ias}, the borders between subdivisions were drawn as straight parallel lines. For this we rely on \textcite{Cintra2013}.
		\item Identification comes from a fuzzy-RDD \parencite{Cattaneo2019,Cattaneo2019a}: the lines drawn in the 1530s subdivide the land into capitan\'ias, some of which return to direct-rule by the Portuguese Crown, while others remain semi-autonomous until the late 18\textsuperscript{th} century. This identification strategy is similar to \textcite{Michalopoulos2014}.
		\item We hypothesise that colonial autonomy had a positive effect on long-run economic development through higher state capacity \parencite{Besley2009}, because capit\~aes had to distribute land more equally (to white settlers) and raise taxes, while the Crown could rely on its coffers to provide public goods. 
		\begin{itemize}
			\item I.e. capit\~aes behaved like \emph{stationary bandits} \`{a} la \textcite{Olson1993}.
			\item This raises the question: are there records of the Crown's investments in Brazil? Or some kind of ``balance of payments'' we could tease out capital flows during the colonial era.
		\end{itemize}
		\item Historical data sources:
		\begin{itemize}
			\item \textcite{Naritomi2007,Naritomi2012}
			\item \textcite{Fujiwara2017}
		\end{itemize}
	\end{itemize}

	\clearpage
	\printbibliography
	
\end{document}