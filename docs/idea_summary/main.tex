\documentclass[11pt,a4paper]{article}

% Bibliography package
\usepackage[backend=biber,
style=authoryear,
date=year,
block=space,
maxbibnames=10,
doi=false,isbn=false,url=true,eprint=false]
{biblatex}

\AtEveryBibitem{%
	\ifentrytype{article}{%
		\clearfield{url}%
		\clearfield{urldate}%
		\clearfield{archivePrefix}%
		\clearfield{arxivId}%
	}{%
		\ifentrytype{misc}{%
			\clearfield{url}%
			\clearfield{urldate}%
			\clearfield{archivePrefix}%
			\clearfield{arxivId}%
		}{%
			\ifentrytype{incollection}{%
				\clearfield{url}%
				\clearfield{urldate}%
				\clearfield{archivePrefix}%
				\clearfield{arxivId}%
			}{%
			}%
		}%
	}%
}%

\addbibresource{../Bibliographies/202012.bib}


% Set languange:
\usepackage[utf8]{inputenc}
\usepackage[UKenglish]{babel}
\usepackage{csquotes}

% Math Packages
\usepackage{amsmath}
\usepackage{amsfonts}
\usepackage{amssymb}

% Page layout packages
\usepackage[margin=0.8in, top=0.5in]{geometry}
\usepackage[onehalfspacing]{setspace}
%\usepackage{fancyhdr}
%\setlength{\headheight}{14pt}

% Figure packages
\usepackage{graphicx}
\usepackage{color}
\usepackage{xcolor}

% Table packages
\usepackage[justification=centering]{caption}
\usepackage{subcaption}
\usepackage{longtable}
\usepackage{tabularx}
\usepackage{booktabs} % For Stata Tables
%\usepackage{dcolumn} % to align decimals in tables
\usepackage{siunitx} % to align decimals in tables
\usepackage{rotating}
\usepackage{bigstrut}
\usepackage{multirow}
\usepackage{wrapfig}

% Other packages
\usepackage[pdftex,colorlinks=true]{hyperref}
\hypersetup{pdfauthor={Nicol\'as A. Lillo B. and Felipe Valencia C.},%
	pdftitle={Research Idea - Capitanias},%
	citecolor=blue,%
	urlcolor=purple,%
	filecolor=cyan}
%\usepackage[raggedright]{titlesec} % Avoid hyphenation in section titles

\renewcommand{\and}{ \& }

% Declare title info:
\title{Colonial Autonomy and State-Capacity: Evidence from Brazilian Capitan\'ias}
\date{December 2020}
\author{Nicol\'as A. Lillo B.\thanks{Department of Economics, Pontificia Universidad Javeriana, \emph{Corresponding Author: \href{mailto:nicolas_lillob@javeriana.edu.co}{nicolas\_lillob@javeriana.edu.co}}} \and Felipe Valencia C.\thanks{Vancouver School of Economics, University of British Columbia and IZA}}

\begin{document}
	\maketitle
	
	\section{Sketch of Ideas}
	\begin{itemize}
		\item We plan to answer the question in \textcite{Iyer2010}: what had better (less worse) consequences, direct or indirect rule by a colonizing power.
		\item We will exploit quasi-random variation stemming from the way that Brazil was colonized. When Brazil was subdivided in \emph{capitan\'ias}, the borders between subdivisions were drawn as straight parallel lines. For this we rely on \textcite{Cintra2013}.
		\item Identification comes from a fuzzy-RDD \parencite{Cattaneo2019,Cattaneo2019a}: the lines drawn in the 1530s subdivide the land into capitan\'ias, some of which return to direct-rule by the Portuguese Crown, while others remain semi-autonomous until the late 18\textsuperscript{th} century. This identification strategy is similar to \textcite{Michalopoulos2014}.
		\item We hypothesise that colonial autonomy had a positive effect on long-run economic development through higher state capacity \parencite{Besley2009}, because capit\~aes had to distribute land more equally (to white settlers) and raise taxes, while the Crown could rely on its coffers to provide public goods. 
		\begin{itemize}
			\item I.e. capit\~aes behaved like \emph{stationary bandits} \`{a} la \textcite{Olson1993}.
			\item The argument is similar to Acemoglu's story about the Virginia company \parencite{AcemogluRobinson2012_whynationsfail}. Basically, when people were allowed to settle the land, they were given property rights, which led to better institutions. We need to explore those mechanisms. For Capitan\'ias, because the capit\~ao couldn't exploit the whole land, they had to actually redistribute it to bring more colonists so that they could actually occupy the land. By doing so  you perhaps lowered land inequality and solidified private ownership. Need to explore colonization patterns.
			\item This raises the question: are there records of the Crown's investments in Brazil? Or some kind of ``balance of payments'' we could tease out capital flows during the colonial era.
		\end{itemize}
		\item Historical data sources:
		\begin{itemize}
			\item \textcite{Naritomi2007,Naritomi2012}
			\item \textcite{Fujiwara2017}
		\end{itemize}
	\end{itemize}

	\clearpage
	\section{Other thoughts}
	\begin{itemize}
		\item \textbf{Proxy for indirect rule}: ``\textcite{Lange2009} constructs proxies of indirect	rule in 1955 for 33 British colonies in Africa and elsewhere without major European settlements with the share of colonially recognized customary law cases to the total share of law cases brought in courts and then examines the association of this variable with democracy, state capacity, and political development in 1997–98 . The analysis shows that democracy, bureaucratic quality, and state capacity are considerably worse in former British colonies where indirect rule was widely practiced, such as Sierra Leone, Malawi, and Nigeria.'' \parencite[61]{Michalopoulos2020}
		\item \textbf{Determinants of direct/indirect rule}: ``As indirect rule correlates with colonizer identity, precolonial political centralization, \parencite{Gerring2011} and geographic-ecological features \parencite{DeJuan2017}, it is quite challenging isolating its impact with cross- country or cross-colony approaches (like the ones of \textcite{Lange2009} and \textcite{Ali2019}). Thus, recent works exploit regional within-colony variability which was far from negligible.'' \parencite[63]{Michalopoulos2020}
		\item \textbf{\emph{Causal} Effects of direct/indirect rule}: ``\textcite{Lechler2018} look at Namibia, an interesting case because the North was ruled via local chiefs, while the German and South African administration ruled directly the Southern provinces. While German colonizers had initially planned to extend power and settle in the more developed North, a rinderpest pandemic killed around 95 percent of cattle herds in the South, ``freeing'' territory in central and southern provinces. The pandemic altered the strategy providing some quasi-random variation. In 1905 the Reichstag passed a resolution stating that policing in South West Africa should be limited to regions of direct German interest. Soon afterwards the colonial administration put a veterinary cor- don fence and in 1907 it institutionalized a formal police boundary separating ``white'' and ``black'' South West Africa. \textcite{Lechler2018} compare political beliefs and participation using Afrobarometer surveys on the two sides of the colonial border. The regression discontinuity estimates uncover sharp differences at the border; pro-democracy beliefs and electoral participation are considerably lower on the north- ern side of the colonial border that was ruled indirectly via despotic chiefs. Moreover, support for authority and traditional leaders is higher on the northern side of the border. These results that stem from a credible empirical design offer direct support to Mamdani’s thesis that by empowering local chiefs, indirect rule promoted authoritarianism postindependence.'' \parencite[64]{Michalopoulos2020}
		\item \textbf{\emph{Causal} Effects of concessions}: ``\textcite{Lowes2020} examine the legacy of these strategies focusing on two rubber concessions in northern Congo, the Anglo-Belgian India Rubber Company (ABIR) and Anversoise. Employing a spatial regression discontinuity design that compares outcomes at the border of the concessions, the authors explore their impact on various contemporary outcomes. They uncover several interesting patterns. First, living conditions, as reflected in the DHS composite wealth index (which takes into account access to basic public goods, quality of housing, and ownership of assets) are lower in areas just inside of the con- cessions. Second, literacy, schooling, vaccination rates, and height are significantly lower inside the historical concessionary boundaries. Third, chiefs’ characteristics differ considerably at the historical border. Chiefs are less likely to be elected by the local community and public good provisions are worse inside the concession borders. Fourth, locals’ respect for authority is significantly higher in villages that were affected by forced labor. Fifth, individuals residing inside the concession boundaries have higher levels of trust and exhibit higher levels of pro-social behavior, suggesting that social capital was strengthened by the extractive colonial practices and the associated violence. Sixth, similar pat- terns emerge when the authors examine development outcomes across all concessions in Congo Free State. Their results are important, not only because they look at the most exemplifying example of colonial oppression, but because these concessions did minimal (if any) investment, and, hence, the estimates isolate the extractive nature of these policies.'' \parencite[64]{Michalopoulos2020}
	\end{itemize}
	\clearpage
	\printbibliography
	
\end{document}